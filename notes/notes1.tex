\documentclass[11pt]{article}
\usepackage{amsmath, amssymb, amsfonts, slashed, dsfont, hyperref, comment, feynmp}
\usepackage{xcolor}
\usepackage{url}
\usepackage[margin=1in]{geometry}
\usepackage{sectsty}
\sectionfont{\large}

% Needed for feynmp
\DeclareGraphicsRule{*}{mps}{*}{}

\renewcommand\u[1]{\text{#1}} % No space
\newcommand{\us}[1]{\text{ #1}} % Space
\newcommand{\ee}[1]{\times10^{{#1}}}
\renewcommand\L{\mathcal{L}}
\newcommand{\M}{\mathcal{M}}
\renewcommand\d{\partial}
\newcommand{\bigo}{\mathcal{O}}
\newcommand{\tr}{\operatorname{tr}}
\newcommand{\vev}[1]{\langle{#1}\rangle}
\newcommand{\diag}{\operatorname{diag}}
\newcommand{\sgn}{\operatorname{sgn}}

\begin{document}

\section{Wavy heliospheric current sheet}

Both components of the drift velocity are modified by the waviness:
\begin{itemize}
    \item
        There is a Heaviside step function in the Parker field:
        \begin{align*}
            \vec{B} &= A_c B_0 \left( \frac{r_0}{r} \right)^2 \left( \hat{r} - \frac{\Omega r \sin\theta}{V_{sw}} \hat{\phi} \right) H(\theta' - \theta),
        \end{align*}
        where $\theta'$ is the angular extent of the HCS.  In Strauss et al 2012 this is given by eq 11:
        \begin{align*}
            \theta' &= \frac{\pi}{2} + \sin^{-1}\left[ \sin\alpha \sin\left( \phi - \phi_0(t) + \frac{\Omega r}{V_{sw}} \right) \right],
        \end{align*}
        where $\phi_0(t) = \phi_0 + \Omega t$ and $\phi_0$ is an arbitrary phase.  In the code, the step function appears when calculating the gradient/curvature part of the drift velocity.

        \textbf{TODO}:
        \begin{itemize}
            \item
                How is $\phi_0$ determined?
            \item
                $\alpha$ should also be time dependent.  How should I implement this? Parameter file should probably point to table of $\alpha$ vs $t$, which the parameter object can then parse.  The date at which the particle was observed at Earth can be specified in the run file.
        \end{itemize}
    \item
        The HCS drift velocity is also affected.  It takes the form (eq 17)
        \begin{align*}
            \vec{v}_{hcs} &= v_{hcs} \left[ \cos A_\xi \xi \sin \Psi \hat{r} + \sin A_\xi \xi \hat{\theta} + \cos A_\xi \xi \cos\Psi \hat{\phi} \right] A_c \sgn q.
        \end{align*}
        The angle $\xi$ ($\beta$ in the reference) between the tangent to the HCS and the radial line passing through the point on the HCS is given by
        \begin{align*}
            \tan^2 \xi &= \left[ -r \frac{\d\theta'}{\d r} \right]^2 + \left[ \frac{1}{\sin\theta} \frac{\d \theta'}{\d\phi} \right]^2\\
            \implies \tan \xi &= \frac{\Omega r}{V_{sw}} \frac{1}{\sin \Psi} \frac{\sqrt{\sin^2\alpha - \cos^2\theta'}}{\sin\theta'} \us{(for Parker field)}.
        \end{align*}
        The sign of $\xi$, $A_\xi = \pm 1$, is equal to the sign of $\d\theta'/\d r$:
        \begin{align*}
            A_\xi &= \sgn \cos\left( \phi - \phi_0(t) + \frac{\Omega r}{V_{sw}} \right).
        \end{align*}
        $v_{hcs}$ is given by the usual approximation:
        \begin{align*}
            v_{hcs} &= \left[ 0.457 - 0.412 \frac{L}{r_L} + 0.915 \frac{L^2}{r_L^2} \right] v,
        \end{align*}
        where $L$ is the smallest distance from the particle's position to the HCS.  If $L > 2 r_L$, $v_{hcs}$ is zero.  Finally, $\xi$ is evaluated at the HCS point which minimizes $L$.
\end{itemize}

The hard part of these modifications is calculating $L$, which cannot be done analytically and is expensive.  If
\begin{itemize}
    \item
        $|\theta - \frac{\pi}{2}| \leq \alpha$,
    \item
        $\theta < \frac{\pi}{2} - \alpha$ and $L_+^{th} \leq 2 r_L$, where $L_+^{th} = r \cos(\alpha + \theta)$ is the distance from the particle to the surface bounding the HCS above,
    \item
        Or $\theta > \frac{\pi}{2} + \alpha$ and $L_-^{th} = - r \cos(\theta - \alpha) \leq 2 r_L$,
\end{itemize}
we compute $L$ using the Nelder-Mead simplex algorithm.  Here are the steps for minimizing $L(r_s, \phi_s)$, where $r_s$ and $\phi_s$ are the coordinates of a point in the sheet:
\begin{enumerate}
    \setcounter{enumi}{-1}
    \item
        Choose an initial set of points $\vec{x}_1$, $\vec{x}_2$, $\vec{x}_3$.
    \item
        Order vertices: $L(\vec{x}_1) \leq L(\vec{x}_2) \leq L(\vec{x}_3)$.
    \item
        Compute $\vec{x}_O = \frac{1}{2} (\vec{x}_1 + \vec{x}_2)$.
    \item
        Compute reflected point $\vec{x}_R = \vec{x}_O + \alpha (\vec{x}_O - \vec{x}_3)$.
        
        If $L(\vec{x}_1) \leq L(\vec{x}_R) < L(\vec{x}_2)$, replace $\vec{x}_3$ with $\vec{x}_R$ and go to step 1.
    \item
        If $L(\vec{x}_R) < L(\vec{x}_1)$, compute expanded point $\vec{x}_E = \vec{x}_O + \gamma (\vec{x}_R - \vec{x}_O)$.  

        \quad If $L(\vec{x}_E) < L(\vec{x}_R)$, replace $\vec{x}_3$ with $\vec{x}_E$ and go to step 1.

        \quad Else, replace $\vec{x}_3$ with $\vec{x}_R$ and go to step 1.

        Else, go to step 5.
    \item
        Compute contracted point $\vec{x}_C = \vec{x}_O + \rho (\vec{x}_3 - \vec{x}_O)$.

        If $L(\vec{x}_C) < L(\vec{x}_3)$, replace $\vec{x}_3$ with $\vec{x}_C$ and go to step 1.

        Else, go to step 6.
    \item
        For all but the $\vec{x}_1$, replace $\vec{x}_i$ with $\vec{x}_1 + \sigma (\vec{x}_i - \vec{x}_1)$.  Go to step 1.
\end{enumerate}
Standard values for the constants are $\alpha = 1$, $\gamma = 2$, $\rho = 1/2$, $\sigma = 1/2$.  

The only tough part of this is choosing the initial set of points.  Since $\theta'$ is ``periodic'' in $r \to r + 4 \frac{\Omega}{V_{sw}}$,
\begin{align*}
    \left\{ \left( r - 2 \frac{\Omega}{V_{sw}}, \phi \right), \left( r + 2 \frac{\Omega}{V_{sw}}, \phi \right), \left( r, \phi + \frac{\pi}{6} \right) \right\}
\end{align*}
seem like good guesses.  \textbf{This will need to be tested, though.}

\end{document}
