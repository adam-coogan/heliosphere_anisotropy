\documentclass[11pt]{article}
\usepackage{amsmath, amssymb, amsfonts, slashed, dsfont, hyperref, comment, feynmp}
\usepackage{xcolor}
\usepackage{url}
\usepackage[margin=1in]{geometry}
\usepackage{sectsty}
\sectionfont{\large}

% Needed for feynmp
\DeclareGraphicsRule{*}{mps}{*}{}

\renewcommand\u[1]{\text{#1}} % No space
\newcommand{\us}[1]{\text{ #1}} % Space
\newcommand{\ee}[1]{\times10^{{#1}}}
\renewcommand\L{\mathcal{L}}
\newcommand{\M}{\mathcal{M}}
\renewcommand\d{\partial}
\newcommand{\bigo}{\mathcal{O}}
\newcommand{\tr}{\operatorname{tr}}
\newcommand{\vev}[1]{\langle{#1}\rangle}
\newcommand{\diag}{\operatorname{diag}}
\newcommand{\sgn}{\operatorname{sgn}}

\begin{document}

\section{The anisotropy}

Let $N(e, \vec{r}, t)$ be the local cosmic ray number density per unit energy.  Then the anisotropy $\Delta$ is (see arXiv:astro-ph/0308470, equation 14)
\begin{align*}
    \Delta &\equiv \frac{N_f - N_b}{N_f + N_b} = \frac{3 D}{c} \frac{\nabla N}{N},
\end{align*}
which is called the dipolar or streaming anisotropy.  The RHS of this equation only applies when the anisotropy is small.  This is also how the anisotropy is calculated in heliosphere simulations.

\section{Pulsar anisotropy at the heliopause}

See arXiv:0905.0636.  In Appendix A it is shown that the $N$ for electrons and positrons from a pulsar is
\begin{align*}
    N(E, \vec{r}, t) &= \frac{Q_0}{\pi^{3/2} r_{\u{diff}}^3} (1 - E / E_{\u{max}})^{\Gamma - 2} \left( \frac{E}{1\us{GeV}} \right)^{-\Gamma} e^{-\frac{E}{(1 - E / E_{\u{max}}) E_{\u{cut}}}} e^{-(r / r_{\u{diff}}(E))^2},
\end{align*}
where
\begin{align*}
    r_{\u{diff}} (E, t) &\approx 2\sqrt{D(E) (t - t_0) \frac{1 - (1 - E / E_{\u{max}}(t))^{1 - \delta}}{(1 - \delta) E / E_{\u{max}}(t)}},\\
    E_{\u{max}}(t) &= \frac{1}{b_0(t - t_0)}.
\end{align*}
See the pulsar\_spectrum notebook for parameter values.

\end{document}
